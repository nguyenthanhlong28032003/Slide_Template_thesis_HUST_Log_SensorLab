\subsection{Bộ tham số tán xạ trong đánh giá mạch cao tần}

\begin{frame}{Bộ tham số tán xạ trong đánh giá mạch cao tần}

\begin{itemize}
    \item Bộ tham số tán xạ \(S\):
\end{itemize}
\begin{equation}
    \left[
    \begin{array}{c}
         V_1^- \\
         V_2^- \\
         \vdots \\
         V_N^-
    \end{array}
    \right]
    =
    \left[
    \begin{array}{cccc}
        S_{11} & S_{12} & \ldots & S_{1N} \\
        S_{21} & S_{22} & \ldots & S_{2N} \\
        \vdots & \vdots & \ddots & \vdots \\
        S_{N1} & S_{N2} & \ldots & S_{NN}
    \end{array}\right]
    =
    \left[
    \begin{array}{c}
         V_1^+ \\
         V_2^+ \\
         \vdots \\
         V_N^+
    \end{array}
    \right]
\end{equation}
hay viết đơn giản:
\begin{equation}
    S_{ij} = \dfrac{V_i^-}{V_j^+} |_{V_k^+ = 0, k \neq j}
\end{equation}
Trong đó,
\begin{itemize}
    \item \(V_{i}^+\) là biên độ sóng điện áp tới cổng \(i\), \(V_{i}^-\) là biên độ sóng phản xạ từ cổng \(i\).
    \item \(S_{ii}\) là hệ số sóng phản xạ tại cổng \(i\), \(S_{ij}\) là hệ số truyền qua từ cổng \(i\) sang cổng \(j\).
\end{itemize}
\end{frame}

